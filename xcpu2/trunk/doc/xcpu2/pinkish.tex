\subsection{The Pinkish development cluster}

Pinkish is the 91-node cluster the system's research team at LANL uses for
development and testing. The compute nodes use coreboot~\cite{coreboot}
instead of a standard BIOS and load the OS kernel and system image over the
Myrinet network using XGet~\cite{xget}. The root filesystem on the compute
nodes contains only few system commands (implemented by busybox), shared
libraries and xcpufs and xcpufs2 daemons.

In addition to the head and compute nodes, we have a number of other
servers connected to the same network running various Linux distributions.
XCPU2 allows us to quickly simulate a particular configuration on all
compute nodes and test for any scalability issues of the software we are
working on. Normally we use a namespace that mounts the distribution's root
filesystem via XCPU2's cache file server. The user directories are mounted
from pinkish head node over NFS and we use tmpfs for temporary files on the
nodes.

This configuration is what was used to obtain the performance results
in the next section.
