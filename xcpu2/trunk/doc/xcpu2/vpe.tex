%
% I realized that I may get flak for talking about this because it was vaguely
% product, but never really materialized.  Fortunately the customer talked about it
% http://www.prace-project.eu/hpc-training/prace-winter-school/Carteni_PRACE_WS.pdf
% So we should probably cite that
%


\subsection{Virtual�PowerXcell�Environment�(VPE) Hypernodes}

Roadrunner~\cite{roadrunner}, the world's first petascale supercomputer, was 
architected as a hybrid system composed of general purpose Opteron processors 
with Cell-based accelerators.  
Roadrunner's success led to interest in smaller scale and more loosely coupled
hybrid clusters.
The Virtual PowerXcell Environment (VPE) was the underlying runtime of a prototype
hybrid blade cluster made up of hypernodes.  Each hypernode consisted of a Power 6 
blade (JS22) acting as general purpose front-end to a number of back-end PowerXcell 
blades (QS22) acting as compute accelerators connected over a high-performance 
Infiniband network.

The intent of the hypernode systems software environment was to maintain the
illusion of a tightly coupled single system even though the applications were running 
on a loosely coupled cluster.  In order to accomplish this we specially tagged hybrid 
applications with a custom header identifying them as VPE executables.  
We then built a custom binfmt~\cite{binfmt} module which would recognize the 
VPE header and transparently reserve cluster resources using a workload scheduler 
and use Xcpu2 to schedule the execution of the application on the remote node 
using the front-end nodes environment and operating system.  With the console 
redirected to the same terminal which initiated the execution, environment
variables duplicated and signals forwarded, the user is unaware that the 
application is even running on a remote system.  
The nature of this execution model is such that there is a one-to-one correlation
between the proxy application on the front-end node and the instance running on the
back-end cluster node.  The implication is that we can't take much advantage of the
treespawn or caching mechanisms provided by XCPU2 when fanning out computation to
many back-end nodes.
This transparent remote execution mechanism was originally developed as part of the 
PROSE LibraryOS project~\cite{prose}~\cite{libra}.

The hypernode file system environment was maintained by a GPFS~\cite{gpfs} cluster which
was mounted on the front-end JS22 nodes.
The PowerXcell blades were configured as diskless drone systems
and booted with a minimal ramdisk containing only enough resources to establish
themselves as Xcpu2 servers.  All application executables and data were obtained
by the QS22s via the 9P connection to the JS22 front-end established by Xcpu2.
This configuration had the effect of aggregating file system access by the QS22's 
through the JS22 (in much the same way BlueGene compute nodes aggregate file
access through their I/O nodes~\cite{bgp}) allowing a more simple configuration 
environment,  and an extremely small systems software footprint on the QS22 nodes.

One unforeseen complication of our transparent distributed execution model
was in dealing with intelligent runtimes who optimize their communication
based on thread location.  For example, when we ran MPI applications in a
VPE environment we could convince MPI to start extra threads to be scheduled
on the remote nodes, but since it thought those threads were running locally it
would try to use shared memory for communication between them.  Since
memory is not one of the resources shared between front-end and back-end nodes
this caused our naive configuration of the MPI application to fail.  With some
effort we were able to identify an MPI library version and configuration which
would allow us to override the shared memory communication optimizations.

Another issue we had was trying to do proper resource allocation of the back-end
accelerators.  Each QS22 PowerXcell blade had 16 synergistic processing units~\cite{ibm-cell}
(SPUs) which applications could use.  Because these accelerator resources were not
accounted for via the Xcpu2 file server, we had no integrated mechanism to monitor
or reserve SPUs for a particular applications and had to resort to a third-party resource 
reservation system which the applications had to interact with directly.  We will discuss
some ideas of how to address this in the future work section.
